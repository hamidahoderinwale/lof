% Options for packages loaded elsewhere
\PassOptionsToPackage{unicode}{hyperref}
\PassOptionsToPackage{hyphens}{url}
\PassOptionsToPackage{dvipsnames,svgnames,x11names}{xcolor}
%
\documentclass[
]{article}

\usepackage{amsmath,amssymb}
\usepackage{iftex}
\ifPDFTeX
  \usepackage[T1]{fontenc}
  \usepackage[utf8]{inputenc}
  \usepackage{textcomp} % provide euro and other symbols
\else % if luatex or xetex
  \usepackage{unicode-math}
  \defaultfontfeatures{Scale=MatchLowercase}
  \defaultfontfeatures[\rmfamily]{Ligatures=TeX,Scale=1}
\fi
\usepackage{lmodern}
\ifPDFTeX\else  
    % xetex/luatex font selection
\fi
% Use upquote if available, for straight quotes in verbatim environments
\IfFileExists{upquote.sty}{\usepackage{upquote}}{}
\IfFileExists{microtype.sty}{% use microtype if available
  \usepackage[]{microtype}
  \UseMicrotypeSet[protrusion]{basicmath} % disable protrusion for tt fonts
}{}
\makeatletter
\@ifundefined{KOMAClassName}{% if non-KOMA class
  \IfFileExists{parskip.sty}{%
    \usepackage{parskip}
  }{% else
    \setlength{\parindent}{0pt}
    \setlength{\parskip}{6pt plus 2pt minus 1pt}}
}{% if KOMA class
  \KOMAoptions{parskip=half}}
\makeatother
\usepackage{xcolor}
\setlength{\emergencystretch}{3em} % prevent overfull lines
\setcounter{secnumdepth}{5}
% Make \paragraph and \subparagraph free-standing
\makeatletter
\ifx\paragraph\undefined\else
  \let\oldparagraph\paragraph
  \renewcommand{\paragraph}{
    \@ifstar
      \xxxParagraphStar
      \xxxParagraphNoStar
  }
  \newcommand{\xxxParagraphStar}[1]{\oldparagraph*{#1}\mbox{}}
  \newcommand{\xxxParagraphNoStar}[1]{\oldparagraph{#1}\mbox{}}
\fi
\ifx\subparagraph\undefined\else
  \let\oldsubparagraph\subparagraph
  \renewcommand{\subparagraph}{
    \@ifstar
      \xxxSubParagraphStar
      \xxxSubParagraphNoStar
  }
  \newcommand{\xxxSubParagraphStar}[1]{\oldsubparagraph*{#1}\mbox{}}
  \newcommand{\xxxSubParagraphNoStar}[1]{\oldsubparagraph{#1}\mbox{}}
\fi
\makeatother


\providecommand{\tightlist}{%
  \setlength{\itemsep}{0pt}\setlength{\parskip}{0pt}}\usepackage{longtable,booktabs,array}
\usepackage{calc} % for calculating minipage widths
% Correct order of tables after \paragraph or \subparagraph
\usepackage{etoolbox}
\makeatletter
\patchcmd\longtable{\par}{\if@noskipsec\mbox{}\fi\par}{}{}
\makeatother
% Allow footnotes in longtable head/foot
\IfFileExists{footnotehyper.sty}{\usepackage{footnotehyper}}{\usepackage{footnote}}
\makesavenoteenv{longtable}
\usepackage{graphicx}
\makeatletter
\newsavebox\pandoc@box
\newcommand*\pandocbounded[1]{% scales image to fit in text height/width
  \sbox\pandoc@box{#1}%
  \Gscale@div\@tempa{\textheight}{\dimexpr\ht\pandoc@box+\dp\pandoc@box\relax}%
  \Gscale@div\@tempb{\linewidth}{\wd\pandoc@box}%
  \ifdim\@tempb\p@<\@tempa\p@\let\@tempa\@tempb\fi% select the smaller of both
  \ifdim\@tempa\p@<\p@\scalebox{\@tempa}{\usebox\pandoc@box}%
  \else\usebox{\pandoc@box}%
  \fi%
}
% Set default figure placement to htbp
\def\fps@figure{htbp}
\makeatother

\makeatletter
\@ifpackageloaded{caption}{}{\usepackage{caption}}
\AtBeginDocument{%
\ifdefined\contentsname
  \renewcommand*\contentsname{Table of contents}
\else
  \newcommand\contentsname{Table of contents}
\fi
\ifdefined\listfigurename
  \renewcommand*\listfigurename{List of Figures}
\else
  \newcommand\listfigurename{List of Figures}
\fi
\ifdefined\listtablename
  \renewcommand*\listtablename{List of Tables}
\else
  \newcommand\listtablename{List of Tables}
\fi
\ifdefined\figurename
  \renewcommand*\figurename{Figure}
\else
  \newcommand\figurename{Figure}
\fi
\ifdefined\tablename
  \renewcommand*\tablename{Table}
\else
  \newcommand\tablename{Table}
\fi
}
\@ifpackageloaded{float}{}{\usepackage{float}}
\floatstyle{ruled}
\@ifundefined{c@chapter}{\newfloat{codelisting}{h}{lop}}{\newfloat{codelisting}{h}{lop}[chapter]}
\floatname{codelisting}{Listing}
\newcommand*\listoflistings{\listof{codelisting}{List of Listings}}
\makeatother
\makeatletter
\makeatother
\makeatletter
\@ifpackageloaded{caption}{}{\usepackage{caption}}
\@ifpackageloaded{subcaption}{}{\usepackage{subcaption}}
\makeatother

\usepackage{bookmark}

\IfFileExists{xurl.sty}{\usepackage{xurl}}{} % add URL line breaks if available
\urlstyle{same} % disable monospaced font for URLs
\hypersetup{
  pdftitle={Procedural Knowledge Libraries for Data Visualization Engineers},
  pdfkeywords={procedural knowledge, data
visualization, metadata, knowledge management},
  colorlinks=true,
  linkcolor={blue},
  filecolor={Maroon},
  citecolor={Blue},
  urlcolor={Blue},
  pdfcreator={LaTeX via pandoc}}


\title{Procedural Knowledge Libraries for Data Visualization Engineers}
\author{}
\date{}

\begin{document}
\maketitle
\begin{abstract}
This work explores the concept of procedural knowledge libraries
tailored for data visualization engineers. It discusses the dynamic
nature of such libraries, the importance of capturing procedural
metadata, and the implications for knowledge preservation and retrieval.
The paper also examines existing tools and methodologies, highlighting
their limitations and proposing a framework for more effective
procedural knowledge management.
\end{abstract}

\renewcommand*\contentsname{Table of Contents}
{
\hypersetup{linkcolor=}
\setcounter{tocdepth}{2}
\tableofcontents
}

\section{Introduction}\label{introduction}

Initially, this centered around procedural knowledge as a whole entity
and now it's information about the process of building new knowledge
itself. Explicating these processes makes it possible to audit
{[}human/machine{]} reasoners. You can imagine having repos of
procedural knowledge for those analyzing (building models of) the same
dataset. This framework would then make it possible to assess the
mechanistic differences between their approaches, clearly defining the
branches of thought that took place.

The important distinction to make here is that libraries are systems;
they are dynamic, as the contents within them are living. To date,
traditional ``libraries'' have assumed otherwise. Furthermore, the
objects inside them are defined as an artifact of knowledge. That is,
artifacts that are the product of thought and developed with a goal in
mind. The function of the system is then to be a home to these artifacts
and preserve them for prosperity's sake, but more importantly for
retrieval's sake. As such, while the integrity of the work should be
preserved, there are ways to build the system and store the objects such
that this can be done.

Provenance refers to the lineage of an object. Provenance, synonymous
with metadata, has been developed with the static knowledge artifact in
mind, so extending upon it for the sake of the dynamic artifact is key
for a ``new age of libraries.''

Now, we are faced with the question of what it means to develop new
knowledge. That is, what is something tractable than we can anchor to,
to help build better libraries: libraries that work for knowledge that
evolves over time. I use the instance of model-building. While there's
no dedicated library to models, they exist within research papers,
repositories, and notebooks. Here, I outline the ways that we can
capture how they are developed (e.g.~improved version control,
collaboratively, integrating ``layered context architecture''), what
they mean (i.e.~to help people find exactly what they need by
explicating what models and their requests mean), what it looks like to
retrieve them (the ideal interface to help people search in accordance
with their preferences), etc. And then I consider how they may fit into
a larger system: how they may be cataloged, organized (e.g., the
metadata that can be assigned to them, scalable metadata attribution,
etc.), and stored (e.g.~internet search engine that works for
``findings.'').

Here, we focus on the model-builder who is building graphical models or
visualizations. Visualization is often the last step in conveying
insight and analysis of data that has been obtained through the research
process. Therefore, within this test-bed, we lay out a framework and
suggest the limitations and trade-offs in real-world application (i.e.
Log-based CRDTs being computationally expensive than traditional CRDTS),
concluding with discussion on promising speculative and existing areas
of work to address them.

\section{Status quo}\label{status-quo}

\begin{itemize}
\tightlist
\item
  File versioning where you decide what's a version. For example, naming
  versions of a project
  \href{https://researchdata.wisc.edu/news/version-control-for-research-projects/}{as}
  ``v1, v2, v3.'' Most document libraries have some version of crude
  file records. \href{https://arxiv.org/}{arXiv}, the preprint
  repository will keep versions of a paper as they're re-uploaded with
  edits from the authors. Google Scholar indexes the versions of a paper
  and will link it to the same citation.
\item
  Versioning with a Distributed Version Control System (DVCS) such as
  Git where you have logs of your past commits (that make your versions)
  and cloud DVCS where you're able to collaborate with others. Here,
  there's not only versions pertinent to the changes that you make, but
  the changes that other people make to your instance of the project,
  forked versions of your project, and the changes that their
  collaborators make to those instances.
\item
  There's a number of domain-specific projects and tools, such as
  \href{https://codalab.org/}{CodaLab} where machine learning
  researchers can host and run their experiments (reproducible) and
  share their papers in executable form as worksheets.
\item
  For visualization workflows--the primary application of this
  project--there's
  \href{https://www.vistrails.org//index.php/Main_Page}{VisTrails}, a
  provenance tracker for visualizations where data is stored in XML
  files. With it, users can query workflows, as the system tracks
  decisions and data products to help answer the actions that led to a
  result. It's core focus is reproducibility. It's mainly developed for
  Python and Python-based visualization libraries.
\item
  There's \href{https://github.com/davidad/chit}{Chit}, an early-stage
  structured data version control project.
\end{itemize}

\href{https://www.perplexity.ai/search/overview-of-viztrails-fCsottRfQR6P0aoMyMLVZg}{Perplexity
trail} (could VizTrail work for a UMAP model and viz? What are its
capabilities?) \# Libraries and why they're valuable

\section{Overview}\label{overview}

\begin{itemize}
\item
  Broadening the scope of the library

  \begin{itemize}
  \item
    Procedural knowledge libraries are living and they're archives of
    ``dynamic'' reasoning.
  \item
    There should not be a distinction because even static artifacts take
    time to make and evolve over time. {Even if we have static artifacts
    our archives should be dynamic to account for development.}
  \item
    Preserving the right to ``priviledged information.''
  \item
    Trails are now included in the corpus of knowledge and librarians
    should be able to navigate them and they should be included in the
    set of things that we want to query to answer the questions that we
    pose. In other words, including the drafted states means you can
    have ``end-to-end'' libraries: libraries of reasoning and libraries
    of artifacts.
  \item
    This won't come without intervention, but then how do you translate
    the ``what'' and ``how'' of development into the ``why?" Next, why
    does it matter if you can do that?
  \item
    Barriers to tracking history

    \begin{itemize}
    \item
      Problems with CRDTs

      \begin{itemize}
      \item
        Eg-walker algorithm as alternative:
        https://arxiv.org/pdf/2409.14252 (allows for fine-grained
        editing history, mentioned the promise in being extended to
        rich-text, graphics, and other applications; HN thread suggests
        that this would need an additoinal CRDT implemented for
        rich-test)
      \item
        (From thread) might be fruitful to look at what exists for
        photoshop and similar software: where layers can cam be
        inserted/deleted and as a result the index of following items
        update. And then there's a question of how layer re-orders would
        work. (a set of operations, even graph where each event
        corresponds to an operation, replay, apply/retreat/advance
        methods for eddicient replay)
      \item
        If you have an advanced CRDT say XCRDT which represents the
        internal structure in your work then there are way ways to
        combine the original operations with the CRDT operations.
      \end{itemize}
    \item
      The non-linearity of history (Log-based CRDTs:
      https://sites.cs.ucsb.edu/~ckrintz/papers/ic2e22.pdf)
    \item
      Format / interoperabilityof historical data
    \item
      Key parameters of queries/history/metadata
    \end{itemize}

    \begin{itemize}
    \tightlist
    \item
      This is useful because you can distinguish goals and deviate from
      these goals (set out by others or your past self) in pointed,
      clear ways.
    \end{itemize}
  \end{itemize}
\item
  The implications of broadening the scope:

  \begin{itemize}
  \item
    It takes more to train to become a librarian. The scope of the
    librarian also evolves. The librarian is always the human but
    machines can be tools to help us become better ones (work faster,
    distributively, and more effectively)
  \item
    Engineering and research questions that stem from this, especially
    on the implementation side.
  \item
    The different layers of context: action-layer, procedural-layer,
    narrative-layer, object/output-layer (reconstructing by laying these
    on top of each other and have them span the space of time in the
    horizontal direction) --\textgreater{} assuming this takes the form
    of a database

    \begin{itemize}
    \item
      From CGPT: ``You have 4-D GithHub (model = code (action) +
      reason/method/strategy (procedural intent, step in readable
      language, what someone else could theoretically carry out) +
      interpretation (why the action was done, what outcome was inferred
      from the reason for the code, reason for the strategy)compared to
      2-D GitHub (action (what was done) + object (the output))''

      \begin{itemize}
      \item
        4-D (new): Model = Action + Rationale + Interpretation + Object
      \item
        2-D (old): Model = Action + Object
      \item
        \textbf{Reconstructing narratives from the procedural
        information: turning procedural information into knowledge}
      \end{itemize}
    \end{itemize}
  \end{itemize}
\end{itemize}

\section{How to read this work. Listing the organization of the
paper.}\label{how-to-read-this-work.-listing-the-organization-of-the-paper.}

\section{Structure and points}\label{structure-and-points}

\begin{itemize}
\item
  Redefine what is a library in terms of its function and its form

  \begin{itemize}
  \tightlist
  \item
    This assumes that there's some core, instrinsic properties that make
    it a library and the rest of the variables we can play with
    according to what's nee
  \end{itemize}
\item
  There's a gap in the way that we currently store knowledge

  \begin{itemize}
  \tightlist
  \item
    What provenance do we already
    \href{https://ieeexplore.ieee.org/stamp/stamp.jsp?tp=&arnumber=7192714}{capture}?
  \end{itemize}
\item
  How we manage information might matter more than just the sheer amount
  of knowledge that we have
\item
  Procedural knowledge is the one we try the least to preserve but it
  holds a wealth of insight
\item
  How do you interact with this kind of procedural knowledge as a
  librarian?

  \begin{itemize}
  \tightlist
  \item
    That is, how do you query and what does retrieval look like? (This
    seems unexplored).
  \end{itemize}

  {Trying to quantify the current (tacit) knowledge loss when you don't
  log}
\end{itemize}

\section{Motivations \& Real-World
applications}\label{motivations-real-world-applications}

\begin{itemize}
\item
  More than just capturing procedural knowledge how do you make it
  operational.
\item
  Being able to have ``structured procedural metadata'\,' for search in
  internet research libraries. For example, you could have
  methodology-based queries. For a tool like Elicit, it would be
  possible to search for papers that use specific methods, had margins
  or metric results within these ranges, and so on: this would make it
  easier to conduct meta-analyses.
\item
  Being able to abstract away the implementation completely and pick
  between different workflows on the output-layer.
\item
  Procedural knowledge libraries are effectively tutorials--they allow
  people to do the things you've done beyond authentication.
\item
  We can get better understandings of systems and do summary statistics
  and make opinions and judgements about the world.
\item
  Help interpreters understand how the ``formalism'' and/or
  ``representation'' disagrees with reality. And being able to assert
  this at a more granular, generalizable angle. Models are not
  objectively worse than each other (for the most part) but they do
  carry different goals and assumptions and it's useful to make that
  clear.
\item
  Increased use of ``unsupervised methods.'' This
  \href{https://arxiv.org/pdf/2111.15506}{paper} mentions ``unsupervised
  explanatory steps; others off-load the analysis to an unsupervised
  model (e.g.~Unsupervised Dimensionality Component Analysis).
\item
  The Colab Data Science Agent isn't very good. There isn't training
  data for this kind of work.
\item
  There's Julius AI, Tableau, Causal (Taimur), etc.
\item
  There's no data analysis reasoning examples that would help them
  express the different choices that they went down.
\item
  What would it look like to have branching paths interface for
  automated reasoners and knowledge work by machines. ``Choose your own
  adventure but for prompting."
\item
  ``Collective intelligence'' in the spirit of the ``Paradigms of
  Intelligence'' team at
  \href{https://www.theatlantic.com/sponsored/google/beyond-the-brain/3944/}{Google}.
  Problems are ``solved'' by small groups or single researchers but
  they're ability to do so is the product of collective human output.
  {What would it look like to work with people who are not in our
  immediate vicinity or take advantage of past knowledge in a way that
  is just as beneficial.}

  \begin{itemize}
  \item
    Solving ``open problems'' can be reframed as a
    \href{https://arxiv.org/pdf/1801.02965}{cooperative search game}.

    \begin{itemize}
    \item
      ``Cooperative search games are collective tasks where all agents
      share the same goal of reaching a target in the shortest time
      while limiting energy expenditure and avoiding collisions.''

      \begin{itemize}
      \tightlist
      \item
        ``The substance to which searchers respond acts as a memory over
        which agents share information about the environment.
        \textbf{The actions of writing, erasing, and forgetting are
        equivalent to production, consumption, and degradation of
        chemoattractant.}
      \end{itemize}
    \end{itemize}
  \end{itemize}
\item
  You can have
  ``\href{https://www.theatlantic.com/sponsored/google/beyond-the-brain/3944/}{super-intelligence}''
  by tapping into ``collective intelligence,'' just by virtue of having
  the power of the sum. It then becomes a question of having the various
  human inputs be as synergistic as possible, enabling coordination,
  encouraging participation, etc.
\end{itemize}

\section{Data Visualization and Modelling as a
test-bed}\label{data-visualization-and-modelling-as-a-test-bed}

Scoping this project was important because there are many workflows to
document. I wanted to choose to document the work of someone whose
procedural knowledge is quite abstruse, scarce, and unexplicated, yet
important and ubiquitous. Most knowledge workers are to some degree
faced with the have to make sense of high-dimensional, complex data.
Thus, here, I focus on procedural knowledge libraries for the data
visualizer.

Data visualization is a way of simplifying the data, drawing patterns,
and then visualizing them in a way that can be understood by others.
Still, there are many decisions that a data visualization engineer will
make and it is unclear why they make them.

For a given dataset, a visualization engineer may follow a number of
paths, all of which are reasonable in their own ways. However, a
consumer of a visualization often lacks the context to make sense of the
decisions that were made during the development process. This leads to
the interpreter taking conclusions for granted. Arguably, this is the
root of lacking media literacy--an important problem in today's society.

Here, we will try to explain why procedural knowledge in this case is
valuable, the kinds that are valuable to capture, and how it may be
retrieved, organized, and cataloged. I call this a library of procedural
knowledge for the visualization engineer, and thus use this as an
opportunity to derive a functional definition of the contemporary
library.

I try to describe the features of the data visualization engineer's
workflow as a basis for coming up with a more extensible framework for
procedural knowledge libraries that could work for a number of them. I
frame the answer to these questions in terms of rough workflow sketches.
Thus, this work also serves as the basis of conversation for what the
ideal library for this kind of knowledge work, recognizing that there
are many details that have been omitted out of brevity and ignorance.

Data analysis is hard to ``learn.'' There's a lot of tacit knowledge and
``intuitions'' that are built over time which makes transparency
difficult. There's tacit judgements, exploratory detours, and aesthetic
and communication trade-offs.

There's unique features of graphical model-building: \textbf{we will
explore how to diff non-text artifacts}; we will learn the limits of
formalizing and abstraction.

Extrapolating to anytime you want to make intermediate states explicit.

For visualization methods, there's no ``absolute consistency'' for what
is a saliency map compared to a heatmap or neural activation. Procedural
knowledge can help us intuit what are better explanations for a given
model. Theory-guided data science.

Visualizations of learned representations or models. Visualization are
post-hoc interpretations and we render visualization to qualitatively
capture what models have learned.

\section{Relevant Work}\label{relevant-work}

https://learn.hex.tech/docs/explore-data/projects/history-and-versions
(Hex Notebooks and their versioning capabilities)

Is it possible to come up with a
\href{https://dl.acm.org/doi/pdf/10.1145/3236386.3241340}{rigorous
standard of correctness}?

Overview of (ML) model viz libs - Pysvelte - LLM Comporator:
https://arxiv.org/pdf/2402.10524

\section{Relevant}\label{relevant}

Margo Seltzer

New York Times R\&D

Case-studies on Datawrapper, Tableau, etc.

\subsection{Annotated Bib}\label{annotated-bib}

\begin{itemize}
\item
  \href{https://mbaradad.github.io/learning_with_noise/}{Learning to See
  by Looking at Noise}
\item
  \href{https://graphics.cs.wisc.edu/Papers/2016/Gle16/compjournal.pdf}{A
  Framework for Considering Comprehensibility in Modelling}
\item
  In this
  \href{https://liorpachter.wordpress.com/2024/02/26/all-of-us-failed/}{All
  of Us failed} post (From UMAP blog posts). A grife of the author is
  that there were no descriptions: in the Rye paper there was little
  justification provided for the decisions that were made such as
  picking 16 principal components or \emph{What the difference would be
  between 20 principal components and 16}. And there is no ``general
  analysis describing the robustness of results {[}of the{]}
  parameter.'' The author questions why the entirety of the ``All of
  Us'' consortium chose to use UMAP.
\item
  \href{UMAP\%20paper}{https://arxiv.org/pdf/1802.03426v3}

  \begin{itemize}
  \tightlist
  \item
    The paper mentions that ``fuzzy topological representation'' is a
    way to ``merge the incompatible local views of the data.''
  \end{itemize}
\item
  https://arxiv.org/pdf/2111.15506 (Towards a comprehensive
  visualization of structure in data

  \begin{itemize}
  \item
    Data transformations as described in the paper: 1) Take non-linear
    manifold in lower-dim where a visualization would be largely
    uninformative 2) then you take linear projections of the high-dim
    data and make it human-readable (2 or 3-d)
  \item
    The problem is with \textbf{non-linear methods} which are
    computationally complex and less deterministic (?). Examples of such
    methods include t-SNE and UMAP.
  \item
    They address this with (standardized?) parametric configs? They
    claim this would be generalizable to other non-linear methods.
  \item
    There's commonly trade-offs with capturing global structure compared
    to local structure. They propose a ``retrieval information
    approach'' where's the neighbour retriever visualizer (NeRV) that
    looks at the cost of precision relative to recall. They do this in
    terms of retrieving/missing neighbors in the high-dim representation
    and the low-dim representation.
  \item
    You're also trading off speed compared to accuracy.

    \begin{itemize}
    \tightlist
    \item
      I assume the premise is that \textbf{chunking, discreteness, and
      cleaner parameters} translate into better queries compared to more
      continuous data.
    \end{itemize}
  \end{itemize}
\item
  https://www.pnas.org/doi/epdf/10.1073/pnas.95.25.14863)
\item
  https://alarmingdevelopment.org/?p=1570
\item
  https://www.tableau.com/sites/default/files/2023-01/2008-GraphicalHistories-InfoVis.pdf
  Graphical Histories for Visualization: Supporting Analysis,
  Communication, and Evaluation (2008)
\item
  https://link.springer.com/article/10.1186/1471-2288-10-14
  (Understanding human functioning using graphical models)
\end{itemize}

\section{Why a library of (tacit) knowledge is valuable for the model
developer}\label{why-a-library-of-tacit-knowledge-is-valuable-for-the-model-developer}

It seems odd that you can't get a snapshot of the entire state of your
model as time goes on.

For typed models, you have the equational representation and the
diagrammatic representation.

We want to assess the relationships between different variables and then
manipulate our graphs to help us best understand what those
relationships are.

With Datawrapper you can link ``live'' datasets and it would be useful
to see how the graphs change in relaton to changes in the dataset.

\href{https://nytlabs.com/projects/chronicle.html}{This} NYT project
looked at changes in language use in different articles over time. There
are semantic tags for the words that appear the most frequently.

\section{Kinds of Procedural Knowledge to
Capture}\label{kinds-of-procedural-knowledge-to-capture}

\begin{itemize}
\item
  Making sense of the dataset (i.e.~whether it is tabular, etc.)
  Modifying the dataset based on the features that wll be abstracted
  (coming up with columns, etc.).
\item
  Establishing a relation to try and synthesize a graphical design.
  Making sense of structural properties (with the domain sets) ``and
  their
  \href{https://dl.acm.org/doi/pdf/10.1145/22949.22950?curius=2438}{functional
  relationships}.''
\item
  Explaining why a data instance is anomalous. Explaining why an
  instance was an anomalous and defining the anomaly
  itself.https://dl.acm.org/doi/10.1145/3609333 (this work mentions
  visual models for anamoly detection)
\item
  the assumptions
\item
  Choosing the features to focus on for the lower-dim representation of
  the dataset.
\item
  Assuming some pattern
\item
  Trying to validate the pattern
\item
  Iterating, debugging the visual representation
\end{itemize}

\section{Benefits}\label{benefits}

\begin{itemize}
\tightlist
\item
  Explainability (understanding how the fnal visual came-to-be)
\end{itemize}

\section{Analogies and Relevant
Concepts}\label{analogies-and-relevant-concepts}

\begin{itemize}
\item
  Cognitive trails (temporal qualities, defining ``memories'' in the
  space of your memory)

  \begin{itemize}
  \item
    There's causal links, conditional branches and attentional focus.
    \textbf{There's temporal sequences of thought. Showing the different
    \emph{considerations} and when they arise, the different , and try
    to explicate many implicit processes.}
  \item
    you reason about how ``understanding of a domain manifests in an
    agent's behavior'' (RL-related, behavioral psychology, markovian
    game theory, etc.)
  \item
    ``having understandable \textbf{stories} of reasoning''
  \end{itemize}

  \begin{itemize}
  \item
    How do you convert understanding Newton's laws of motion to
    calculating a projectile (Claude-generated example)
  \item
    Tool selection (i.e.~picking the right statistical theory based on
    understanding of probability theory (Claude-generated).
  \item
    condition-action
  \end{itemize}
\item
  What do CRDT logs look like?

  \begin{itemize}
  \item
    How do states relate over time?
  \item
    \textbf{\emph{Can you reconstruct a reasoning process from CRDT
    logs?}}

    \begin{itemize}
    \item
      CRDTs don't keep a full operational log. They only store the
      current state. Logs (for auditing, provenance, replay) then you
      can have operational logs manually made or through a layered
      system.

      \begin{itemize}
      \item
        having a directed versus undirected graph
      \item
        JSON patches
      \item
        Track the intent and target of every change
      \item
        you can replay them in any order (but how does work when
        decisions only make sense in certain/a given sequence(s))
      \item
        here is where branching timelines could fit in
      \end{itemize}

      \begin{itemize}
      \item
        *ChatGPT says: ``This preserves the integrity of insight while
        still getting the power of distributed sync."
      \item
        \textbf{CRDTs are built to ensure convergence; but you can
        add-on a ``procedural layer'' for the sake of interpretablity.}
      \end{itemize}
    \end{itemize}
  \item
    Synthesis (\href{https://core.ac.uk/download/pdf/29195125.pdf}{of
    visual environments})
  \end{itemize}
\item
  Coresets

  \begin{itemize}
  \item
    Partitioning datasets that upon reconstruction preserve all the
    properties of the full dataset.
  \item
    It's partitioned to speed up computations, making approximations
    faster (e.g.~clustering, regression), and ``compressing'' procedural
    and data-intensive pipelines.
  \item
    The structure is a weighted subset for a given objective function.
  \item
    With each small subset of data you have the key properties of the
    transformed dataset . The coreset s ``stored'' with the step where
    it was computed (\textbf{linked list and indexing})
  \item
    The goal is to have a \textbf{\emph{`}'miniature, audit-ready trace}
  \item
    Ordered, annotated log as a form of ``narrative layer''

    \begin{itemize}
    \tightlist
    \item
      You can have human + machine annotations versus. algorithmically
      derived narrative layers
    \end{itemize}
  \end{itemize}
\item
  Record linkage
\end{itemize}

\section{Case-studies}\label{case-studies}

\section{Version Control}\label{version-control}

Meaningful diffs when you're working in different languages.

JS is both more imperative and un-typed. How do you retrieve the goals
of different states when you're using these languages? This is compared
to when I use SQL and I have clear explications of my goals and then if
I migrate and there's an error then I can infer that the next steps are
bug-fixing.




\end{document}
